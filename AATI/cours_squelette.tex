\documentclass[11pt]{article}
\usepackage[french]{babel}

\usepackage[utf8]{inputenc}
\usepackage{palatino}
\usepackage[T1]{fontenc}


\usepackage{url}
\usepackage{amsmath}

\usepackage[top=2cm,bottom=2cm,left=2.1cm,right=2.1cm,headsep=10pt,a4paper]{geometry}
\usepackage{fancyhdr}


\usepackage{graphicx,float} % figure et placement de figure
\usepackage{listings} %%inclusion de programmes

\lstset{
language=C++,
basicstyle=\ttfamily\small, %
identifierstyle=\color{black}, %
keywordstyle=\color{blue}, %
stringstyle=\color{blue}, %
commentstyle=\it\color{green}, %
}

\usepackage{xcolor}

\pagestyle{fancy}
\lhead{}
\chead{\fontsize{10}{10}{M2IM - UCBL - 2014/2015}}
\rhead{}

\lfoot{}
\cfoot{\thepage}
\rfoot{}


\renewcommand{\headrulewidth}{0pt}
\renewcommand{\footrulewidth}{0pt}

 \author{\fontsize{14}{14}{Aurélien CHEMIER 10908892}}
 \title{\fontsize{16}{16}{Analyse et Traitement d'Image}\\ {Squelettes}}
 \date{\fontsize{11}{11}{\today}}

\begin{document}

\thispagestyle{empty}
\maketitle

\newpage
\tableofcontents
\newpage

\section{Approches générales, Objectifs, Intérêts}

	\subsection{Qu'est ce qu'un squelette ?}
		C'est une représentation filiforme, \textit{centrée} dans la forme à analyser, représentative de son \textit{allure générale}
		(élongation, déformation, composantes connexes ...).

	\subsection{Intérêts}
		\begin{itemize}
		\item C'est une version simplifiée d'un objet, tout en ayant la même \textit{homotopie} que l'objet initial.
		\item Il peut éventuellement permettre l'isolation des \textit{des composantes connexes} d'un objet.
		\item Il permet un gain de place mémoire (condense l'information bidimensionnel en une représentation linéaire).
		\end{itemize}

		La notion de squelette est apparue pour l'étude d'objets minces (En effets, pour de telles figures, c'est l'allure d'une représentation filiforme qui est importante) (reconnaissance de caractères).

	\subsection{Concept de \textit{boules maximales} (Calabi)}
		C'est une définition plus formelle de la notion de squelette, à la fois dans le cas continu et le cas discret.

		Calabi est le premier à prouver que le squelette par boules maximales (ou plus exactement la \textit{fonction d'étanchéité} permet de reconstruire l'ensemble initial).

\section{Historique}
	\begin{itemize}
	\item Blum (1961) introduit le concept du squelette (\textit{axe médian} ou \textit{axe symétriques}). \\
	Sa définition formalise la notion intuitive de squelette (représentation minimale d'un ensemble sous forme de ligne d'épaisseur 1).
	\item Il utilise le concept du \textit{feu de prairie}.
	\end{itemize}

	\subsection{Définition}
		En supposant qu'un feu se propage dans l'ensemble X considéré à vitesse uniforme à partir des contours de celui-ci, l'axe médian (squelette) de X est défini comme l'ensemble des points d'intersections des différents fronts de feu.

	\subsection{Introduction de propriétés}
		\begin{itemize}
		\item Le squelette est \textit{mince}, formé de l'union d'arcs et de courbes.
		\item Il n'est pas affecté par des transformations de types translation ou rotation.
		\end{itemize}

\section{Approche du squelette par boules maximales}
	\subsection{Définition, notation}
		Soit $A$ un ensemble non vide, fermé et borné dans $R^2$, on établit que toute boulé fermé contenue dans $A$ est elle-même contenue dans une boule maximale incluse dans $A$.

		Pour toute partie A  fermée bornée non vide de $R^2$, on appelle squelette de A $S(A)$ l'ensemble des centres des boules maximales contenue dans A.

		\textbf{Remarque :} Les boules maximales de  $A$ touchent $\delta A$ en au moins 2 points distincts.






\end{document}

%\begin{figure}[H]
%      \centering
 %     \includegraphics[scale=0.7]{Image/grandTableau.png} 
 %     \caption{Position de départ}
%      \label{fig:grandTableau}
%  \end{figure}

%	\[
%	\begin{pmatrix}
%	1 & 2 & 3 \\
%	4 & 5 & 6 \\
%	7 & 8 & 9
%	\end{pmatrix}
%	\]
